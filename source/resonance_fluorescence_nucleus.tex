    % This file is part of nrf_seminar.

    % nrf_seminar is free software: you can redistribute it and/or modify
    % it under the terms of the GNU General Public License as published by
    % the Free Software Foundation, either version 3 of the License, or
    % (at your option) any later version.

    % nrf_seminar is distributed in the hope that it will be useful,
    % but WITHOUT ANY WARRANTY; without even the implied warranty of
    % MERCHANTABILITY or FITNESS FOR A PARTICULAR PURPOSE.  See the
    % GNU General Public License for more details.

    % You should have received a copy of the GNU General Public License
    % along with nrf_seminar.  If not, see <https://www.gnu.org/licenses/>.

\def \SLIDESTART {1}
\def \SLIDEPHOTONSOURCE {2}
\def \SLIDERADIOACTIVE {3}
\def \SLIDEDECAY {4}
\def \SLIDEEMISSION {5}
\def \SLIDERECOIL {6}
\def \SLIDEWHEEL {7}
\def \SLIDEMOESSBAUER {8}
\def \SLIDEBEAM {9}
\def \SLIDEDETECTION {10}

\begin{textblock}{15.}(0., -3.5)
    \begin{tikzpicture}
        %%% Definitions
        %% Mathematical constants
        \def \NWAVEFORMSAMPLES {50}

        %% Incoming EM wave
        \def \AMPLITUDE {0.4}
        \def \WAVEFORMLENGTH {2.}
        \def \WAVELENGTH {0.4}
        \def \LONGWAVELENGTH {0.6}

        \def \BEAMSTART {0.}
        \def \AMPLITUDEBEAM {0.6}
        \def \WAVEFORMLENGTHBEAM {4.}
        \def \WAVELENGTHBEAMLOW {0.3}
        \def \WAVELENGTHBEAMHIGH {0.5}

        %% Nucleus
        \def \NUCLEUSRADIUS {0.6}

        \def \TARGETNUCLEUSX {5.5}
        \coordinate (TARGETNUCLEUSPOSITION) at (\TARGETNUCLEUSX, 0.);

        \def \SOURCENUCLEUSX {1.5}
        \coordinate (SOURCENUCLEUSPOSITION) at (\SOURCENUCLEUSX, 0.);

        \def \STATEX {2*\NUCLEUSRADIUS}
        \def \MOTHERGROUNDSTATEY {-1.5}
        \def \DAUGHTEREXCITEDSTATEY {-2.2}
        \def \DAUGHTERGROUNDSTATEY {-3.8}

        %% Wheel
        \def \WHEELRADIUS {1.}

        %%% Drawing
        %% Target nucleus
        \visible<-\SLIDEBEAM>{
            \draw [very thick, black, dashed] (TARGETNUCLEUSPOSITION) circle [radius = \NUCLEUSRADIUS];
        }

        % Level scheme of daughter nucleus
        \visible<\SLIDERADIOACTIVE-\SLIDEMOESSBAUER>{
            \draw [very thick, black] (\TARGETNUCLEUSX, \DAUGHTEREXCITEDSTATEY) -- (\TARGETNUCLEUSX + \STATEX, \DAUGHTEREXCITEDSTATEY);
            \draw [very thick, black] (\TARGETNUCLEUSX, \DAUGHTERGROUNDSTATEY) -- (\TARGETNUCLEUSX + \STATEX, \DAUGHTERGROUNDSTATEY);
            \draw (\TARGETNUCLEUSX + 0.5*\STATEX, \DAUGHTERGROUNDSTATEY) node[anchor=north] {$^{57}$Fe};

        %% Source nucleus
            \draw [very thick, black, dashed] (SOURCENUCLEUSPOSITION) circle [radius = \NUCLEUSRADIUS];
        }
        % Recoil direction
        \visible<\SLIDERECOIL>{
            \draw [->, very thick, black] ($ (SOURCENUCLEUSPOSITION) - \NUCLEUSRADIUS*(1.,0.)$) -- ($ (SOURCENUCLEUSPOSITION) - 2*\NUCLEUSRADIUS*(1.,0.)$);
        }

        % Level scheme of mother nucleus
        \visible<\SLIDERADIOACTIVE-\SLIDEMOESSBAUER>{
            \draw [very thick, black] (\SOURCENUCLEUSX - 2*\NUCLEUSRADIUS, \MOTHERGROUNDSTATEY) -- (\SOURCENUCLEUSX - 2*\NUCLEUSRADIUS + \STATEX, \MOTHERGROUNDSTATEY);
            \draw (\SOURCENUCLEUSX + 0.5*\STATEX, \DAUGHTERGROUNDSTATEY) node[anchor=north] {$^{57}$Fe};

        % Level scheme of daughter nucleus
            \draw [very thick, black] (\SOURCENUCLEUSX, \DAUGHTEREXCITEDSTATEY) -- (\SOURCENUCLEUSX + \STATEX, \DAUGHTEREXCITEDSTATEY);
            \draw [very thick, black] (\SOURCENUCLEUSX, \DAUGHTERGROUNDSTATEY) -- (\SOURCENUCLEUSX + \STATEX, \DAUGHTERGROUNDSTATEY);
            \draw (\SOURCENUCLEUSX - 0.5*\STATEX, \DAUGHTERGROUNDSTATEY) node[anchor=north] {$^{57}$Co};
        }

        % Transitions
        % Electron capture
        \visible<\SLIDEDECAY-\SLIDEMOESSBAUER>{
            \draw [->, very thick, black] (\SOURCENUCLEUSX - 0.5*\STATEX, \MOTHERGROUNDSTATEY) -- (\SOURCENUCLEUSX + 0.5*\STATEX, \DAUGHTEREXCITEDSTATEY);
        }
        \visible<\SLIDEEMISSION-\SLIDEMOESSBAUER>{
        % Decay to ground state
            \draw [->, very thick, black] (\SOURCENUCLEUSX + 0.5*\STATEX, \DAUGHTEREXCITEDSTATEY) -- (\SOURCENUCLEUSX + 0.5*\STATEX, \DAUGHTERGROUNDSTATEY);
        
        % Emitted photon
            \draw [very thick, \ORANGE, domain=\SOURCENUCLEUSX+0.5*\STATEX:\SOURCENUCLEUSX+0.5*\STATEX+\WAVEFORMLENGTH, samples=\NWAVEFORMSAMPLES] plot (\x, {\AMPLITUDE*sin(\WAVEFORMLENGTH/\WAVELENGTH*2*pi/4*\x r) + \DAUGHTERGROUNDSTATEY + 0.5*(\DAUGHTEREXCITEDSTATEY - \DAUGHTERGROUNDSTATEY)});
        }
        \visible<\SLIDERECOIL>{
            \draw [very thick, \RED, domain=\SOURCENUCLEUSX+0.5*\STATEX:\SOURCENUCLEUSX+0.5*\STATEX+\WAVEFORMLENGTH, samples=\NWAVEFORMSAMPLES] plot (\x, {\AMPLITUDE*sin(\WAVEFORMLENGTH/\LONGWAVELENGTH*2*pi/4*\x r)});
        }
        \visible<\SLIDEWHEEL-\SLIDEMOESSBAUER>{
            \draw [very thick, \ORANGE, domain=\SOURCENUCLEUSX+0.5*\STATEX:\SOURCENUCLEUSX+0.5*\STATEX+\WAVEFORMLENGTH, samples=\NWAVEFORMSAMPLES] plot (\x, {\AMPLITUDE*sin(\WAVEFORMLENGTH/\WAVELENGTH*2*pi/4*\x r)});
        }

        %% Wheel
        \visible<\SLIDEWHEEL>{
            \draw [very thick, black] ($ (SOURCENUCLEUSPOSITION) + \WHEELRADIUS*(0,1) + \NUCLEUSRADIUS*(0,1)$) circle [radius = \WHEELRADIUS];
            \draw [->, very thick, black] ($ (SOURCENUCLEUSPOSITION) + \WHEELRADIUS*(0,1) + \NUCLEUSRADIUS*(0,1) + 0.5*\WHEELRADIUS*(1., 0.)$) arc [start angle = 0, end angle = 250, radius = 0.5*\WHEELRADIUS];
        }

        \visible<\SLIDEMOESSBAUER>{
            %% Neighboring atoms
            % Top left atom
            \draw [very thick, black, dashed] ($ (SOURCENUCLEUSPOSITION) + 3*\NUCLEUSRADIUS*(0,1)$) circle [radius = \NUCLEUSRADIUS];

            % Top right atom
            \draw [very thick, black, dashed] ($ (SOURCENUCLEUSPOSITION) + 3*\NUCLEUSRADIUS*(1,1)$) circle [radius = \NUCLEUSRADIUS];

            % Bottom right atom
            \draw [very thick, black, dashed] ($ (SOURCENUCLEUSPOSITION) + 3*\NUCLEUSRADIUS*(1,0)$) circle [radius = \NUCLEUSRADIUS];

            % Connection bottom left to top left
            \draw [very thick, black] ($ (SOURCENUCLEUSPOSITION) + \NUCLEUSRADIUS*(0,1)$) -- ($ (SOURCENUCLEUSPOSITION) + 2*\NUCLEUSRADIUS*(0,1)$);

            % Connection top left to top right
            \draw [very thick, black] ($ (SOURCENUCLEUSPOSITION) + 3*\NUCLEUSRADIUS*(0,1) + \NUCLEUSRADIUS*(1,0)$) -- ($ (SOURCENUCLEUSPOSITION) + 3*\NUCLEUSRADIUS*(0,1) + 2*\NUCLEUSRADIUS*(1,0)$);

            % Connection top right to bottom right
            \draw [very thick, black] ($ (SOURCENUCLEUSPOSITION) + 2*\NUCLEUSRADIUS*(0,1) + 3*\NUCLEUSRADIUS*(1,0)$) -- ($ (SOURCENUCLEUSPOSITION) + 1*\NUCLEUSRADIUS*(0,1) + 3*\NUCLEUSRADIUS*(1,0)$);

            % Connection bottom left to bottom right
            \draw [very thick, black] ($ (SOURCENUCLEUSPOSITION) + \NUCLEUSRADIUS*(1,0)$) -- ($ (SOURCENUCLEUSPOSITION) + 2*\NUCLEUSRADIUS*(1,0)$);
        }

        \visible<\SLIDEBEAM>{
            \draw [very thick, \ORANGE, domain=\BEAMSTART:\BEAMSTART+\WAVEFORMLENGTHBEAM, samples=\NWAVEFORMSAMPLES] plot (\x, {\AMPLITUDEBEAM*sin(\WAVEFORMLENGTH/\WAVELENGTH*2*pi/4*\x r)});

            \draw [very thick, \RED, domain=\BEAMSTART:\BEAMSTART+\WAVEFORMLENGTHBEAM, samples=\NWAVEFORMSAMPLES] plot (\x, {\AMPLITUDEBEAM*sin(\WAVEFORMLENGTH/\WAVELENGTHBEAMHIGH*2*pi/4*\x r)});

            \draw [very thick, \YELLOW, domain=\BEAMSTART:\BEAMSTART+\WAVEFORMLENGTHBEAM, samples=\NWAVEFORMSAMPLES] plot (\x, {\AMPLITUDEBEAM*sin(\WAVEFORMLENGTH/\WAVELENGTHBEAMLOW*2*pi/4*\x r)});
        }
    \end{tikzpicture}
\end{textblock}

\begin{textblock}{5.}(7., -6.)
    \begin{center}
        \begin{tabular}{lll}
        & Atom & Nucleus \\
        \hline
        Res. Energy & $\unit[10^{0}]{eV}$ & $\unit[10^{6}]{eV}$ \\
        Res. Width & $\unit[10^{-7}]{eV}$ & $\unit[10^{0}]{eV}$ \\
        System mass $\times c^2$ & $\unit[10^{9}]{eV}$ & $\unit[10^{9}]{eV}$ \\
        \end{tabular}
    \end{center}
\end{textblock}

\begin{textblock}{6.}(8.5, -1.5)
    \begin{itemize}
    \visible<\SLIDEPHOTONSOURCE->{
        \item \textbf{Photon source}: \\ 
        radioactive sources \visible<\SLIDEWHEEL->{[Moon (1951)\visible<\SLIDEMOESSBAUER->{, M\"ossbauer (1958)}]}\visible<\SLIDEBEAM->{, photon beams [Hayward et al. (1957)]}
    }
    \visible<\SLIDEDETECTION>{
        \item \textbf{Photon spectroscopy}: \\
        particle detectors \\(, X-ray diffraction)
    }
    \end{itemize}
\end{textblock}

\begin{textblock}{8.}(0., -5.)
    \visible<\SLIDERECOIL-\SLIDEMOESSBAUER>{
        $\text{relative recoil} \approx \frac{\text{Photon energy}}{\visible<\SLIDEMOESSBAUER>{N \times } \text{Nuclear mass} \times c^2}$
    }
\end{textblock}

\begin{textblock}{6.}(0., -2.)
    \visible<\SLIDEDETECTION>{
        \textit{Realistic image of a gamma-ray detector with some indication of the dimensions.
        I used a picture of two people working on the Gammasphere detectors array, which had been shown in an earlier talk of the series.}
        %\includegraphics[width=\textwidth]{figures/gammasphere.jpg}
    }
\end{textblock}