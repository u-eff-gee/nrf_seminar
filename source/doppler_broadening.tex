    % This file is part of nrf_seminar.

    % nrf_seminar is free software: you can redistribute it and/or modify
    % it under the terms of the GNU General Public License as published by
    % the Free Software Foundation, either version 3 of the License, or
    % (at your option) any later version.

    % nrf_seminar is distributed in the hope that it will be useful,
    % but WITHOUT ANY WARRANTY; without even the implied warranty of
    % MERCHANTABILITY or FITNESS FOR A PARTICULAR PURPOSE.  See the
    % GNU General Public License for more details.

    % You should have received a copy of the GNU General Public License
    % along with nrf_seminar.  If not, see <https://www.gnu.org/licenses/>.

\def \SLIDEABSOLUTEZERO {1}
\def \SLIDELN {2}
\def \SLIDEREST {3}
\def \SLIDEROOM {4}
\def \SLIDECRISTAL {5}

\begin{textblock}{9.}(0., -4.)
    \visible<\SLIDEABSOLUTEZERO>{
        \includegraphics[width=\textwidth]{figures/python/cross_section_absolute_zero.pdf}
    }
\end{textblock}
\begin{textblock}{9.}(0., -4.)
    \visible<\SLIDELN-\SLIDEREST>{
        \includegraphics[width=\textwidth]{figures/python/cross_section_ln2.pdf}
    }
\end{textblock}
\begin{textblock}{9.}(0., -4.)
    \visible<\SLIDEROOM>{
        \includegraphics[width=\textwidth]{figures/python/cross_section_room_temperature.pdf}
    }
\end{textblock}
\begin{textblock}{7.}(0., -5.)
    \visible<\SLIDECRISTAL>{
        \includegraphics[width=\textwidth]{figures/crystal_resonance.pdf}
    }
\end{textblock}
\begin{textblock}{7.}(8., 6.)
    \visible<\SLIDECRISTAL>{
        Connection to condensed-matter physics [Lamb, (1939)]
    }
\end{textblock}

\begin{textblock}{8.}(9., -4.)
    \begin{tikzpicture}
        %%% Definitions
        %% Mathematical constants
        \def \NWAVEFORMSAMPLES {50}

        %% Incoming EM wave
        \def \AMPLITUDE {0.6}
        \def \WAVEFORMLENGTH {2.}
        \def \WAVELENGTH {0.5}
        \def \SHORTWAVELENGTH {0.4}
        \def \LONGWAVELENGTH {0.6}

        %% Inertial frames
        \def \COMY {1.5}
        \def \RESTY {-1.5}
        \def \INERTIALFRAMELABELY {1.}

        %% Nucleus
        \def \NUCLEUSX {4.}
        \coordinate (NUCLEUSCOMPOSITION) at (\NUCLEUSX, \COMY);
        \coordinate (NUCLEUSRESTPOSITION) at (\NUCLEUSX, \RESTY);
        \def \NUCLEUSRADIUS {0.5}

        %%% Drawings
        %% Inertial frame labels
        % COM
        \draw (0.5*\NUCLEUSX, \COMY+\INERTIALFRAMELABELY) node[anchor=south] {Center-of-mass frame};
        % Rest
        \visible<\SLIDEREST->{
            \draw (0.5*\NUCLEUSX, \RESTY+\INERTIALFRAMELABELY) node[anchor=south] {Nucleus-at-rest frame};
        }

        %% Nucleus
        % COM
        \draw [very thick, black, dashed] (NUCLEUSCOMPOSITION) circle [radius=\NUCLEUSRADIUS];
        % Rest
        \visible<\SLIDEREST->{
            \draw [very thick, black, dashed] (NUCLEUSRESTPOSITION) circle [radius=\NUCLEUSRADIUS];
        }

       % Slow movement
        \visible<\SLIDELN-\SLIDEREST>{
            \draw [->, very thick, black] ($ (NUCLEUSCOMPOSITION) - \NUCLEUSRADIUS*(1,0) $) -- ($ (NUCLEUSCOMPOSITION) - 2*\NUCLEUSRADIUS*(1,0) $);
        }
        % Fast movement
        \visible<\SLIDEROOM>{
            \draw [->, very thick, black] ($ (NUCLEUSCOMPOSITION) - \NUCLEUSRADIUS*(1,0) $) -- ($ (NUCLEUSCOMPOSITION) - 3*\NUCLEUSRADIUS*(1,0) $);
        }

        %% Incoming wave
        % COM
        \draw [very thick, \RED, domain=0.:\WAVEFORMLENGTH, samples=\NWAVEFORMSAMPLES] plot (\x, {\AMPLITUDE*sin(\WAVEFORMLENGTH/\LONGWAVELENGTH*2*pi/4*\x r) + \COMY});
        % Rest
        \visible<\SLIDEREST>{
            \draw [very thick, \ORANGE, domain=0.:\WAVEFORMLENGTH, samples=\NWAVEFORMSAMPLES] plot (\x, {\AMPLITUDE*sin(\WAVEFORMLENGTH/\WAVELENGTH*2*pi/4*\x r) + \RESTY});
        }
        \visible<\SLIDEROOM->{
            \draw [very thick, \YELLOW, domain=0.:\WAVEFORMLENGTH, samples=\NWAVEFORMSAMPLES] plot (\x, {\AMPLITUDE*sin(\WAVEFORMLENGTH/\SHORTWAVELENGTH*2*pi/4*\x r) + \RESTY});
        }
    \end{tikzpicture}
\end{textblock}