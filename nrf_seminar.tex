\documentclass{beamer}
\usetheme{Madrid}

\usepackage[showboxes]{textpos}

\makeatletter
\setbeamertemplate{footline}
{
  \leavevmode
  \hbox{%
  \begin{beamercolorbox}[wd=.20\paperwidth,ht=2.25ex,dp=1ex,center]{author in head/foot}%
    \usebeamerfont{author in head/foot}\insertshortauthor
  \end{beamercolorbox}%
  \begin{beamercolorbox}[wd=.30\paperwidth,ht=2.25ex,dp=1ex,center]{title in head/foot}%
    \usebeamerfont{title in head/foot}\insertshorttitle
  \end{beamercolorbox}%
  \begin{beamercolorbox}[wd=.30\paperwidth,ht=2.25ex,dp=1ex,center]{author in head/foot}%
    \usebeamerfont{author in head/foot}{Advances in Physics Seminar}
  \end{beamercolorbox}%
  \begin{beamercolorbox}[wd=.20\paperwidth,ht=2.25ex,dp=1ex,right]{date in head/foot}%
    \usebeamerfont{date in head/foot}\insertdate{}\hspace*{2em}
    \insertframenumber{} / \inserttotalframenumber\hspace*{2ex} 
  \end{beamercolorbox}}%
  \vskip0pt%
}
\makeatother

\title{Nuclear Resonance Fluorescence}
\subtitle{Basics, Applications, and Recent Advances}
\author[U. Friman-Gayer]{Udo Friman-Gayer\inst{1,2}}
\institute{
    \inst{1} Department of Physics and Astronomy, University of North Carolina at Chapel Hill, Chapel Hill, NC \\
    \inst{2} Triangle Universities Nuclear Laboratory, Duke University, Durham, NC
}
\date{Advances in Physics Seminar, 06/18/2020}

\begin{document}

\begin{frame}
    \titlepage
\end{frame}

\begin{frame}
    \frametitle{Outline}
    \tableofcontents
\end{frame}

\section{Introduction}

\subsection{'Scattering' of Photons}

\begin{frame}
    \frametitle{Interaction of Photons with Matter}
    \begin{textblock}{15.}(0., -5.)
        \begin{itemize}
            \item Model of two spheres with spring
            \item Diffraction: Study static structure
            \item Inelastic Scattering: Study static structure and dynamics
            \item Resonances: Study dynamics
        \end{itemize}
    \end{textblock}
\end{frame}

\begin{frame}
    \frametitle{Resonance Fluorescence}
    \begin{textblock}{15.}(0., -5.)
        \begin{itemize}
            \item Definition of resonance fluorescence
            \item Level scheme
        \end{itemize}
    \end{textblock}
\end{frame}

\subsection{Resonance Fluorescence in Atoms and Nuclei}

\begin{frame}
    \frametitle{Resonance Fluorescence in Atoms}
    \begin{textblock}{15.}(0., -5.)
        \begin{itemize}
            \item Bohr model of the atom
            \item Absorption spectrum
            \item Laser
        \end{itemize}
    \end{textblock}
\end{frame}

\begin{frame}
    \frametitle{Resonance Fluorescence in Nuclei}
    \begin{textblock}{15.}(0., -5.)
        \begin{itemize}
            \item How to achieve resonance fluorescence
            \item Recoil problem
            \item Solution of the recoil problem
            \item Artificial photon sources
        \end{itemize}
    \end{textblock}
\end{frame}

\section{Basics}

\subsection{Typical dimensions}

\begin{frame}
    \frametitle{Nuclear Resonances}
    \begin{textblock}{15.}(0., -5.)
        \begin{itemize}
            \item Typical widths, energies
            \item Level density
        \end{itemize}
    \end{textblock}
\end{frame}

\begin{frame}
    \frametitle{Doppler Broadening}
    \begin{textblock}{15.}(0., -5.)
        \begin{itemize}
            \item Typical widths, energies
            \item Level density
            \item The special case of $^{229}$Th
        \end{itemize}
    \end{textblock}
\end{frame}

\subsection{Experiments}

\begin{frame}
    \frametitle{Why Absorption Spectroscopy is not Possible}
    \begin{textblock}{15.}(0., -5.)
        \begin{itemize}
            \item Analogy with normal vision
            \item Schematic spectrum, convolve with realistic resolution
        \end{itemize}
    \end{textblock}
\end{frame}

\begin{frame}
    \frametitle{Schematic experiment}
    \begin{textblock}{15.}(0., -5.)
        \begin{itemize}
            \item Typical setup
            \item Typical spectra
        \end{itemize}
    \end{textblock}
\end{frame}

\begin{frame}
    \frametitle{Photon Sources}
    \begin{textblock}{15.}(0., -5.)
        \begin{itemize}
            \item Bremsstrahlung vs. Quasi-Monochromatic (Comp)
            \item Typical spectra
        \end{itemize}
    \end{textblock}
\end{frame}

\section{Applications}
\begin{frame}
    \frametitle{R-Process Nucleosynthesis}
    \begin{textblock}{15.}(0., -5.)
        \begin{itemize}
            \item Photodisintegration versus neutron capture
            \item Example: Pygmy Dipole Resonance
        \end{itemize}
    \end{textblock}    
\end{frame}
\begin{frame}
    \frametitle{Isotope-selective Scanning}
\end{frame}

\end{document}